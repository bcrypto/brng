\section{Выработка имитовставки в режиме~HMAC}\label{HMAC}

\subsection{Функция хэширования}

Используется блочно-итерационная функция хэширования~$h$ 
с длиной блока~$b$~кратной восьми
и значениями длиной~$2l\leq b$.

\subsection{Входные и выходные данные}

Входными данными алгоритма выработки имитовставки
являются ключ~$K\in\{0,1\}^*$ и сообщение~$X\in\{0,1\}^*$.

Выходными данными является слово $Y\in\{0,1\}^{2l}$~---
имитовставка сообщения~$X$ на ключе~$K$.

\subsection{Константы и переменные}

{\bf Слова $\texttt{ipad}$, $\texttt{opad}$.}
Используются фиксированные 
слова~$\texttt{ipad},\texttt{opad}\in\{0,1\}^{b}$,
составленные из повторенных~$b/8$ раз октетов~$\hex{36}$ и~$\hex{5С}$: 
$\texttt{ipad}=\hex{36}\parallel\hex{36}\parallel\ldots\parallel\hex{36}$,
$\texttt{opad}=\hex{5C}\parallel\hex{5C}\parallel\ldots\parallel\hex{5C}$.

{\bf Переменная~$t$.}
Используется переменная~$t\in\{0,1\}^{b}$.
%
Значение~$t$ должно быть уничтожено после использования.

\subsection{Алгоритм}

Вычисление имитовставки сообщения~$X$ на ключе~$K$ 
состоит в выполнении следующих шагов:
\begin{enumerate}
\item
Если $|K|\leq b$, то 
$t\leftarrow K\parallel 0^{b-|K|}$,
иначе 
$t\leftarrow h(K)\parallel 0^{b-2l}$.
\item
$Y\leftarrow h((t\oplus\texttt{ipad})\parallel X)$.
\item
$Y\leftarrow h((t\oplus\texttt{opad})\parallel Y)$.
\item
Возвратить $Y$.
\end{enumerate}
