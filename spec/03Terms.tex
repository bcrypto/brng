\chapter{Термины и определения}

В настоящем стандарте применяются  
следующие термины с соответствующими определениями:

{\bf \thedefctr~имитовставка}:
Двоичное слово, 
которое определяется по сообщению с использованием ключа 
и служит для контроля целостности и подлинности сообщения.

{\bf \thedefctr~ключ}:
Параметр, который управляет криптографическими 
операциями зашифрования и расшифрования, 
выработки и проверки электронной цифровой подписи, 
генерации псевдослучайных чисел и др.

{\bf \thedefctr~одноразовый пароль}:
Пароль, действие которого ограничено сеансом 
аутентификации или промежутком времени.

{\bf \thedefctr~октет}:
Двоичное слово длины~$8$.

{\bf \thedefctr~пароль}:
Секрет, который способен запомнить (обработать) 
человек и который поэтому может принимать сравнительно небольшое 
число значений.

{\bf \thedefctr~псевдослучайные числа}:
Последовательность элементов, полученная в результате выполнения 
некоторого алгоритма и используемая в конкретном случае 
вместо последовательности случайных чисел.
%
%(ИСО 2382-2 [1]).

{\bf \thedefctr~синхропосылка}:
Открытые входные данные криптографического алгоритма,
которые обеспечивают уникальность результатов 
криптографического преобразования на фиксированном ключе.

{\bf \thedefctr~случайные числа}:
Последовательность элементов, каждый из которых не может быть предсказан
(вычислен) только на основе знания предшествующих 
ему элементов данной последовательности.
%
% (ИСО 2382-2 [1]).

%{\bf \thedefctr~случайное число}:
%число, выбранное из определенного набора чисел таким образом, 
%что каждое число из данного набора может быть выбрано 
%с одинаковой вероятностью (ИСО 2382-2 [1]).

{\bf \thedefctr~сообщение}:
Двоичное слово конечной длины.

{\bf \thedefctr~хэш-значение}:
Двоичное слово фиксированной длины, 
которое определяется по сообщению без использования ключа и 
служит для контроля целостности сообщения и для представления 
сообщения в сжатой форме.

{\bf \thedefctr~хэширование}:
Выработка хэш-значений.

