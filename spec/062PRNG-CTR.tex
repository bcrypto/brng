\section{Генерация псевдослучайных чисел в режиме счетчика}\label{PRNG-CTR}

\subsection{Функция хэширования}

Используется функция хэширования~$h$ со значениями длиной~$2l$.

\subsection{Входные и выходные данные}

Входными данными алгоритма генерации псевдослучайных чисел являются 
натуральное $n$, ключ $K\in\{0,1\}^{2l}$ 
и синхропосылка $S\in\{0,1\}^{2l}$. 
Число~$n$ определяет количество генерируемых псевдослучайных чисел.

Используется дополнительное входное слово $X\in\{0,1\}^{2ln}$.
Слово~$X$ записывается в виде~$X=X_1\parallel X_2\parallel\ldots\parallel X_n$,
где~$X_i\in\{0,1\}^{2l}$~--- дополнительные данные,
которые используются на $i$-й итерации алгоритма.
Слова~$X_i$ могут выбираться произвольным образом, 
в том числе случайным или псевдослучайным методом.
По умолчанию~$X_i=0^{2l}$.

Выходными данными алгоритма является слово~$Y\in\{0,1\}^{2ln}$~---
псевдослучайные числа, полученные на ключе~$K$ при использовании
синхропосылки~$S$ и дополнительных данных~$X$. 
Слово~$Y$ записывается в виде
$Y=Y_1\parallel Y_2\parallel\ldots\parallel Y_n$, 
где~$Y_i\in\{0,1\}^{2l}$.

\subsection{Переменные}

Используются переменные $s, r\in\{0,1\}^{2l}$.
%
Значение~$r$ должно быть уничтожено после использования.

\subsection{Алгоритм}

Генерация псевдослучайных чисел состоит в выполнении следующих шагов:
\begin{enumerate}
\item
$s\leftarrow S$.
\item
$r\leftarrow S\oplus 1^{2l}$.
\item
Для $i=1,2,\ldots,n$ выполнить:
\begin{enumerate}
\item
$Y_i\leftarrow h(K\parallel s\parallel X_i\parallel r)$;
\item
$s\leftarrow s\boxplus \langle 1\rangle_{2l}$;
\item
$r\leftarrow r\oplus Y_i$.
\end{enumerate}
\item
$Y\leftarrow Y_1\parallel Y_2\parallel\ldots\parallel Y_n$.
\item
Возвратить $Y$.
\end{enumerate}

