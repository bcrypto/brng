\clearpage
\begin{thebibliography}{999}
\bibitem{RFC-HMAC}
Krawchuk~H., Bellare~M., Canetti~R.
HMAC: Keyed-Hashing for Message Authentication.
Request for Comments: 2104, 1997\\
{\small (HMAC: ключезависимое хэширование для аутентификации сообщений)}

\bibitem{HOTP}
M'Raihi D., 
Bellare M.,
Hoornaert F.,
Naccache D.,
Ranen O.
HOTP: An HMAC-Based One-Time Password Algorithm.
Request for Comments: 4226, 2005\\
{\small (HOTP: Алгоритм генерации одноразовых паролей на основе HMAC)}

\bibitem{TOTP}
M'Raihi D., 
Machani S.,
Pei M.,
Rudell J.
TOTP: Time-Based One-Time Password Algorithm.
Request for Comments: 6238, 2011\\
{\small (TOTP: Алгоритм генерации одноразовых паролей на основе времени)}

\bibitem{OCRA}
M'Raihi D., 
Rudell J.,
Bajaj S.,
Machani S.,
Naccache D.
OCRA: OATH Challenge-Response Algorithm.
Request for Comments: 6287, 2011\\
{\small (OCRA: Алгоритм OATH типа вызов-ответ)}

\bibitem{RD-PRNG}
РД РБ 07040.1206-2003. 
Руководящий документ Республики Беларусь. 
Автоматизированная система межбанковских расчетов. 
Процедура выработки псевдослучайных данных с использованием 
секретного параметра.~--- 
Мн.: Национальный Банк Республики Беларусь, 2003

\label{LastBib}
\end{thebibliography}

